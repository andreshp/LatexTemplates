%%%%%%%%%%%%%%%%%%%%%%%%%%%%%%%%%%%%%%%%%%%%%%%%%%%%%%%%%%%%%%%%%%%%%%%%%%%%%
% Plantilla de Latex para trabajos de Matemáticas.
%
% Autor: Andrés Herrera Poyatos (https://github.com/andreshp) 
%
% La plantilla se encuentra adaptada al español.
%
%%%%%%%%%%%%%%%%%%%%%%%%%%%%%%%%%%%%%%%%%%%%%%%%%%%%%%%%%%%%%%%%%%%%%%%%%%%%%

%----------------------------------------------------------------------------------------
%	PAQUETES Y OTRAS CONFIGURACIONES DEL DOCUMENTO
%----------------------------------------------------------------------------------------

\RequirePackage[l2tabu, orthodox]{nag}  % Produce un warnig en caso de usar un comando obsoleto.

\documentclass{article}

% Paquetes para el diseño de página:
\usepackage{fancyhdr}                   % Utilizado para hacer títulos propios.
\usepackage{lastpage}                   % Referencia a la última página. Utilizado para el pie de página.
\usepackage{extramarks}                 % Marcas extras. Utilizado en pie de página y cabecera.
\usepackage[parfill]{parskip}           % Crea una nueva línea entre párrafos.
\usepackage[margin=3cm]{geometry}       % Asigna la "geometría" de las páginas.
\usepackage{lipsum}                     % Texto dummy. Quitar en el documento final.

% Fuente utilizada. Elija uno de ellos:
\usepackage{courier}                    % Fuente Courier.
%\usepackage{fourier}                   % Fuente Adobe Utopia.
\usepackage{microtype}                  % Mejora la letra final de cara al lector.

% Paquetes para imágenes:
\usepackage[usenames,dvipsnames]{color} % Permite crear colores propios. Utilizado para el bg de Minted.
\usepackage{graphicx}                   % Utilizado para insertar gráficos.
\usepackage{layout}                     % Introduce una imagen con el diseño de la página. Quitar en el documento.

% Paquetes para matemáticas:                     
\usepackage{amsmath,amsthm,verbatim,amssymb,amsfonts,amscd} % Teoremas, fuentes y símbolos.

% Paquetes para tablas:
\usepackage{booktabs}

% Paquetes para adaptar Látex al Español:
\usepackage[spanish,es-noquoting, es-tabla, es-lcroman]{babel} % Cambia 
\usepackage[utf8]{inputenc}                                    % Permite los acentos.
\selectlanguage{spanish}                                       % Selecciono como lenguaje el Español.

% Estilo de página:
\pagestyle{fancy}                      % fancy
\fancyhf{}                             % Limpia la cabecera y el pie de página.
\geometry{left=3cm,right=3cm,top=3cm,bottom=3cm,headheight=1cm,headsep=0.5cm} % Márgenes y cabecera.

% Espacios en el documento:
\linespread{1.1}                        % Espacio entre líneas.
\setlength\parindent{0pt}               % Selecciona la indentación para cada inicio de párrafo.

% Cabecera del documento:
\renewcommand\headrule{                 % Se ajusta la línea de la cabecera.
\begin{minipage}{1\textwidth}           % Elija una de las siguientes líneas:
%    \hrule width \hsize \kern 1mm \hrule width \hsize height 2pt 
    \hrule width \hsize 
\end{minipage}
}
\lhead{\autor}                          % Parte izquierda.
\chead{}                                % Centro.
\rhead{\titulo}                         % Parte derecha.

% Pie de página del documento:
\renewcommand\footrule{                                 % Se ajusta la línea del pie de página.
\begin{minipage}{1\textwidth}                           % Elija una de las siguientes líneas:
%    \hrule width \hsize height 2pt \kern 1mm \hrule width \hsize   
    \hrule width \hsize   
\end{minipage}\par
}
\lfoot{}                                                 % Parte izquierda.
\cfoot{}                                                 % Centro.
\rfoot{Página\ \thepage\ de\ \protect\pageref{LastPage}} % Parte derecha.

%----------------------------------------------------------------------------------------
%	ESTRUCTURA DEL DOCUMENTO
%----------------------------------------------------------------------------------------

\setcounter{secnumdepth}{2}                   % Se enumeran las secciones con profundidad 2.

%----------------------------------------------------------------------------------------
%	MATEMÁTICAS
%----------------------------------------------------------------------------------------

% Nuevo estilo para teoremas
\newtheoremstyle{theorem-style} % Nombre del estilo
  {3pt}                % Espacio por encima
  {3pt}                % Espacio por debajo
  {\itshape}                   % Fuente del cuerpo
  {}                   % Identación: vacío= sin identación, \parindent = identación del parráfo
  {\bf}                % Fuente para la cabecera
  {.}                  % Puntuación tras la cabecera
  {.5em}               % Espacio tras la cabecera: { } = espacio usal entre palabras, \newline = nueva línea
  {}                   % Especificación de la cabecera (si se deja vaía implica 'normal')

% Nuevo estilo para teoremas
\newtheoremstyle{example-style} % Nombre del estilo
  {3pt}                % Espacio por encima
  {3pt}                % Espacio por debajo
  {}                   % Fuente del cuerpo
  {}                   % Identación: vacío= sin identación, \parindent = identación del parráfo
  {\scshape}                % Fuente para la cabecera
  {:}                  % Puntuación tras la cabecera
  {.5em}               % Espacio tras la cabecera: { } = espacio usal entre palabras, \newline = nueva línea
  {}                   % Especificación de la cabecera (si se deja vaía implica 'normal')

% Teoremas:
\theoremstyle{theorem-style}  % Otras posibilidades: plain (por defecto), definition, remark
\newtheorem{theorem}{Teorema}[section]  % [section] indica que el contador se reinicia cada sección
\newtheorem{corollary}[theorem]{Corolario} % [theorem] indica que comparte el contador con theorem
\newtheorem{lemma}[theorem]{Lema}
\newtheorem{proposition}[theorem]{Proposición}
% Definiciones, notas, conjeturas
\theoremstyle{definition}
\newtheorem{definition}{Definición}[section]
\newtheorem{conjecture}{Conjetura}[section] 
\newtheorem*{note}{Nota} % * indica que no tiene contador
% Ejemplos, ejercicios
\theoremstyle{example-style}
\newtheorem{example}{Ejemplo}[section]
\newtheorem{exercise}{Ejercicio}[section]

%----------------------------------------------------------------------------------------
%	NUEVOS COMANDOS
%----------------------------------------------------------------------------------------

% Portada:
\newcommand{\titulo}{Título}  % Título del trabajo.
\newcommand{\fecha}{1 \ de \ Enero \ de \ 2015}                         % Fecha.
\newcommand{\asignatura}{Latex}                                         % Asignatura.
\newcommand{\autor}{Andrés Herrera Poyatos}                             % Autor.

%----------------------------------------------------------------------------------------
%	PORTADA 
%----------------------------------------------------------------------------------------

\title{                                             % Título
    \vspace{2in}
    \textmd{\textbf{\asignatura \\ \titulo}} \\         % - Nombre del trabajo
    \normalsize\vspace{0.1in}\small{\fecha}  \\         % - Fecha (Arriba) 
    \vspace{3in}
}

\author{\textbf{\autor}}                            % Autor
\date{}                                             % Fecha. Elija entre esta y la del título.

%----------------------------------------------------------------------------------------

\begin{document}

\maketitle

%----------------------------------------------------------------------------------------
%	ÍNDICE
%----------------------------------------------------------------------------------------

% Profundidad del Índice:
%\setcounter{tocdepth}{1}

\newpage
\tableofcontents
\newpage

%----------------------------------------------------------------------------------------
%	Sección 1: Deficiones y teoremas
%----------------------------------------------------------------------------------------

\begin{section}{Matemáticas}

\begin{definition}
 $\LaTeX$ es un sistema de composición de textos, orientado a la creación de documentos 
 escritos que presenten una alta calidad tipográfica. Por sus características y 
 posibilidades, es usado de forma especialmente intensa en la generación de 
 artículos y libros científicos que incluyen, entre otros elementos, expresiones 
 matemáticas.
\end{definition}

Se presentan en esta sección una serie de ejemplos del funcionamiento de la plantilla 
para los comandos relacionados con las matemáticas.

\begin{subsection}{Teoremas, Lemas, Proposiciones y Colorarios}

\begin{theorem}
El número $\sqrt{2}$ es irracional.
\end{theorem}
\begin{proof}
La prueba se realiza por reducción al absurdo. Supongamos que es racional. En tal caso,
se puede escribir $\sqrt{2} = \frac{p}{q}$ con $p,q \in \mathbb{N}$ primos relativos.
De la igualdad anterior se deduce:
$$ 2 = \left(\frac{p}{q}\right)^2 = \frac{p^2}{q^2} \Rightarrow 2 q^2 = p^2 $$
Luego $2$ divide a $p$ por ser primo. Entonces, $p = 2k$ con $k \in \mathbb{N}$.
Se tiene que:
$$ 2 q^2 = p^2 = 4k^2 \Rightarrow q^2 = 2k^2 $$
Análogamente, $2$ divide a $q$. Pero esto contradice que $p$ y $q$ sean primos relativos.
\end{proof}

\begin{corollary}
Existen dos números irracionales $x$, $y$ tales que $x^y$ es racional.
\end{corollary}
\begin{proof}
Consideramos $y = \sqrt 2$. Teniendo en cuenta que $\sqrt 2$ es irracional, si 
${\sqrt 2} ^ {\sqrt 2}$ fuese racional se finaliza la prueba. En caso contrario,
podemos tomar $x = {\sqrt 2} ^ {\sqrt 2}$ teniendo que:
 $$ x ^ y = \left({\sqrt 2}^{\sqrt 2}\right) ^ {\sqrt 2} = {\sqrt 2} ^ {\sqrt 2 \sqrt 2} = {\sqrt 2} ^ 2 = 2 $$ 
\end{proof}

\begin{note}
Es cierto que  ${\sqrt 2} ^ {\sqrt 2}$ es irracional por el Teorema de Gelfond–Schneider.
\end{note}

\end{subsection}

\begin{subsection}{Ejemplos y Ejercicios}

\begin{example}
$\mathbb{R}$ no es homeomorfo a $\mathbb{R}^2$ con la topología usual pues los elementos de 
$\mathbb{R}$ tienen orden de conexión $1$ mientras que los de $\mathbb{R}^2$ tienen orden
de conexión $2$.
\end{example}

\begin{exercise}
Probar que en todo anillo $R$ para todo $a \in R$ tal que $a-1$ es una unidad del anillo y 
$n \in \mathbb{N}$ se tiene
\begin{equation}
\sum^n_{i=0} a^i = \frac{a^{n+1}-1}{a-1}
\end{equation}

\end{exercise}

\end{subsection}

\end{section}

%----------------------------------------------------------------------------------------
%	Sección 2: Otros ejemplos
%----------------------------------------------------------------------------------------
\begin{section}{Otros Ejemplos}

\begin{subsection}{Tablas}

\lipsum[1]

\begin{table}[ht]
\caption{Zonas de Memoria en la arquitectura x86-32}
\kern 1mm                                  % Separa caption de la tabla
\centering \begin{tabular}{@{}llc@{}}
\toprule
\multicolumn{2}{c}{Info}                                            \\
\cmidrule(r){1-2}
\textbf{Zona} & \textbf{Descripción}   & \textbf{Memoria Física}    \\ 
\midrule
ZONE\_DMA     & Soporta DMA            & \textless \ 16MB           \\ 
ZONE\_NORMAL  & Páginas normales       & 16–896MB                   \\
ZONE\_HIGHMEM & Alta memoria           & \textgreater \ 896MB       \\ 
\bottomrule
\end{tabular}
\label{table:memoria-x86-32}
\end{table}

\end{subsection}

\begin{subsection}{Descripción}

\begin{description}
\item[ZONE\_DNA] Contiene páginas que soportan la entrada y salida de datos mediante DMA.
\item[ZONE\_DNA\_32] Es análoga a la anterior pero ls páginas solo son accesibles para
dispositivos de 32-bits.
\item[ZONE\_NORMAL] Esta zona contiene páginas normales y regularmente mapeadas.
\item[ZONE\_HIGHMEM] Esta zona contiene \textit{alta memoria}, que consiste en
 aquellas páginas que son mapeadas con poca frecuencia.
\end{description}

\end{subsection}

\pagebreak

\begin{subsection}{Configuración de la página}
Se muestra la configuración de la página. Eliminarla en el documento final. 
Puede dar problemas con la cabecera y pie de página.

\centering\layout
\end{subsection}

\end{section}

\end{document}