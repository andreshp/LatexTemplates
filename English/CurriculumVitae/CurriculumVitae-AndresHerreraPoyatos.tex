%%%%%%%%%%%%%%%%%%%%%%%%%%%%%%%%%%%%%%%%%%%%%%%%%%%%%%%%%%%%%%%%%%%%%%%%%%%%%
% Curriculum Vitae
%
% Author: Andrés Herrera Poyatos (https://github.com/andreshp) 
%
%%%%%%%%%%%%%%%%%%%%%%%%%%%%%%%%%%%%%%%%%%%%%%%%%%%%%%%%%%%%%%%%%%%%%%%%%%%%%

% The template has been taken from:
% http://www.latextemplates.com/template/moderncv-cv-and-cover-letter
% The original information is the following one:
%%%%%%%%%%%%%%%%%%%%%%%%%%%%%%%%%%%%%%%%%
% "ModernCV" CV and Cover Letter
% LaTeX Template
% Version 1.11 (19/6/14)
%
% This template has been downloaded from:
% http://www.LaTeXTemplates.com
%
% Original author:
% Xavier Danaux (xdanaux@gmail.com)
%
% License:
% CC BY-NC-SA 3.0 (http://creativecommons.org/licenses/by-nc-sa/3.0/)
%
% Important note:
% This template requires the moderncv.cls and .sty files to be in the same 
% directory as this .tex file. These files provide the resume style and themes 
% used for structuring the document.
%
%%%%%%%%%%%%%%%%%%%%%%%%%%%%%%%%%%%%%%%%%

%----------------------------------------------------------------------------------------
%	PACKAGES AND OTHER DOCUMENT CONFIGURATIONS
%----------------------------------------------------------------------------------------

\documentclass[10pt,a4paper,sans]{moderncv} % Font sizes: 10, 11, or 12; paper sizes: a4paper, letterpaper, a5paper, legalpaper, executivepaper or landscape; font families: sans or roman

\moderncvstyle{casual} % CV theme - options include: 'casual' (default), 'classic', 'oldstyle' and 'banking'
\moderncvcolor{blue} % CV color - options include: 'blue' (default), 'orange', 'green', 'red', 'purple', 'grey' and 'black'

% Allow spanish accents
\usepackage[utf8]{inputenc}

\usepackage[scale=0.75]{geometry} % Reduce document margins
%\setlength{\hintscolumnwidth}{3cm} % Uncomment to change the width of the dates column
%\setlength{\makecvtitlenamewidth}{10cm} % For the 'classic' style, uncomment to adjust the width of the space allocated to your name

%-----------------------------------------------------------------------------------------------------
% TABLES
%-----------------------------------------------------------------------------------------------------

\usepackage{array}
\usepackage{multicol}
\usepackage{multirow}
\usepackage{booktabs}
\usepackage{float}

\newcolumntype{L}[1]{>{\raggedright\let\newline\\\arraybackslash\hspace{0pt}}m{#1}}
\newcolumntype{C}[1]{>{\centering\let\newline\\\arraybackslash\hspace{0pt}}m{#1}}
\newcolumntype{R}[1]{>{\raggedleft\let\newline\\\arraybackslash\hspace{0pt}}m{#1}}

%----------------------------------------------------------------------------------------
%	NAME AND CONTACT INFORMATION SECTION
%----------------------------------------------------------------------------------------

\firstname{Andrés} % Your first name
\familyname{Herrera Poyatos} % Your last name

% All information in this block is optional, comment out any lines you don't need
\title{Curriculum Vitae}
\address{La Zubia}{Granada, Spain}
\mobile{+34 680 44 16 06}
\phone{+34 958 59 07 85 }
%\fax{(000) 111 1113}
\email{andreshp9@gmail.com}
\homepage{github.com/andreshp}{github.com/andreshp} % The first argument is the url for the clickable link, the second argument is the url displayed in the template - this allows special characters to be displayed such as the tilde in this example
\extrainfo{Ocupation: Student}
\photo[70pt][0.4pt]{pictures/andreshp.jpg} % The first bracket is the picture height, the second is the thickness of the frame around the picture (0pt for no frame)
%\quote{"A witty and playful quotation" - John Smith}

%----------------------------------------------------------------------------------------

\begin{document}

\makecvtitle % Print the CV title

%----------------------------------------------------------------------------------------
%	PERSONAL INFORMATION
%----------------------------------------------------------------------------------------

\section{Personal Information}

	\begin{itemize}
		\item \textbf{Fist name:} Andrés
		\item \textbf{Last name:} Herrera Poyatos
		\item \textbf{Date of birth:} August 9th, 1995
		\item \textbf{Address: } Nº 48 La Yedra Street, La Zubia, Granada, 18140, Spain
		\item \textbf{Email: } \texttt{andreshp9@gmail.com}
		\item \textbf{Mobile number:} +34 680 44 16 06
	\end{itemize}

%----------------------------------------------------------------------------------------
%	EDUCATION SECTION
%----------------------------------------------------------------------------------------

\section{Education}

	\cventry{2009-2013}{\href{http://thales.cica.es/estalmat/}{Estalmat}}{SAEM Thales}{University of Granada}{Granada, Spain}{A project to detect and stimulate the precocious mathematical talent.}
	
	\cventry{2011-2013}{High School}{IES Trevenque}{La Zubia, Granada, Spain}{}{
		\begin{itemize}
			\item Cum Laude at High School. 
			\item Access to university grade -- 13.63 / 14 (13.63 out of 14).
		\end{itemize}
	}
	
	\cventry{2013-Present}{Double Degree in Computer Science and Mathematics}{University of Granada}{Granada, Spain}{}{
		The double degree lasts 5 years and contains 72 ECTS credits per year.
		The following table summarizes my university grades (each subject's grade is presented in the appendix).
		\begin{center}
			\begin{tabular}{|C{0.25\textwidth}|C{0.2\textwidth}|C{0.2\textwidth}|C{0.2\textwidth}|}
				\hline
				\textbf{Degree}  & \textbf{Number of taken subjects} & \textbf{Average grade} & \textbf{Number of Cum Laude grades} \\ 
				\hline
				Computer Science & 18                                & 9.411 / 10                & 13 \\
				\hline
				Mathematics      & 17                                & 9.835 / 10                & 14 \\
				\hline
			\end{tabular}
		\end{center}
		\kern1mm
	}
	
	\cventry{2014}{\href{http://blogs.unia.es/uniatv/archives/2571}{Practical data science and big data: Knime, R, Hadoop and Mahout tools}}{International University of Andalucía (UNIA)}{Baeza}{Grade 10 / 10}{} 
	
	\cventry{2015}{\href{https://www.coursera.org/account/accomplishments/records/AHz97RsSWEVHpqkY}{R Programming}}{Coursera}{Johns Hopkins University}{Average grade -- 100.0 \%}{Verified Statement with Distiction} 
	
	\cventry{2015}{\href{https://www.coursera.org/maestro/api/certificate/get_certificate?course_id=974416}{Algorithms: Design and Analysis, Part 1}}{Coursera}{Stanford University}{Average grade -- 98.0 \%. Statement of Accomplishment}{} 

%----------------------------------------------------------------------------------------
%	PUBLICATIONS SECTION
%----------------------------------------------------------------------------------------

\section{Publications}

	{\large \textcolor{color1}{Conferences}}

		\begin{itemize}
			\item  Andrés Herrera-Poyatos, Francisco Herrera. \textit{Algoritmo Genético con Diversificación Voraz y Equilibrio entre Exploración y Explotación (Genetic Algorithm with Greedy Diversification and Equilibrium between Exploration and Exploitation)}. 10th Spanish Conference on Metaheuristics, Evolutionary and Bio-inspired Algorithms (\textcolor{color1}{\textit{\href{http://www.eweb.unex.es/eweb/maeb2015/}{MAEB 2015}}}), pp. 9--18, 2015.
		\end{itemize}

		\begin{itemize}
			\item  Andrés Herrera-Poyatos, Francisco Herrera. \textit{Algoritmo Memético Equilibrado con Diversificación Voraz. (Memetic Algorithm with Diversity Equilibrium based on Greedy Diversification)}. 16th National Conference on Artificial Intelligence: 2nd Workshop on Metaheuristics and Evolutionary Algorithms (\textcolor{color1}{\textit{\href{http://simd.albacete.org/caepia15/conferencia/jaem2015/}{JAEM 2015}}}), 2015.
		\end{itemize}

		\begin{itemize}
			\item Daniel Peralta, Andrés Herrera-Poyatos, Francisco Herrera. \textit{Un Estudio sobre el Preprocesamiento para Redes Neuronales Profundas y Aplicación sobre Reconocimiento de Dígitos Manuscritos. (A Study on Data Preprocessing for Deep Neuronal Network and its Application to Handwriting Digit Recognition)}. 17th National Conference on Artificial Intelligence (\textcolor{color1}{\textit{\href{http://www.congresocedi.es/es/caepia}{CAEPIA 2016}}}), 2016.
		\end{itemize}

	{\large \textcolor{color1}{Journals}}

		\begin{itemize}
			\item Andrés Herrera-Poyatos, Francisco Herrera. \textit{Genetic and Memetic Algorithm with Diversity Equilibrium based on Greedy Diversification}. Submitted to Information Sciences. Date of submission: March 25th, 2016.
		\end{itemize}
	

%----------------------------------------------------------------------------------------
%	EXPERIENCE
%----------------------------------------------------------------------------------------

\section{Experience}


	\cventry{\footnotesize September 2015 -- July 2016}{Research training contract on metaheuristics and software development}{Fundación General Universidad de Granada - Empresa}{}{}{Computational Intelligence techniques applied to the development of optimization algorithms for resource scheduling on bus and coach companies.}
	
%----------------------------------------------------------------------------------------
%	AWARDS
%----------------------------------------------------------------------------------------

\section{Awards}

\cvitem{2009}{Top 5 in XXV Thales Mathematics Olympiad - Granada (12 -- 13 years old). Classified for the regional phase in Andalucía.}
\cvitem{2009}{Selected for ESTALMAT - Andalucía, a project to detect and stimulate the precocious mathematical talent.}
\cvitem{2011}{2nd place - I Short Story Competition, "Al borde de lo inconcedible", Villa de la Zubia.}
\cvitem{2012}{1st place - II Short Story Competition, "Al borde de lo inconcedible", Villa de la Zubia.}
\cvitem{2013}{4th place - XXIV Spanish Physics Olympiad - Local Phase in Granada province.}
\cvitem{2013}{1st place - XLIX Spanish Mathematics Olympiad - Local Phase in Granada province.}
\cvitem{2013}{Top 12 - XLIX Spanish Mathematics Olympiad - Regional Phase in Andalucía. Classified for the XLIX Spanish Mathematics Olympiad - National Phase.}
\cvitem{2013}{Honourable Mention to the Best High School Academic Record in La Zubia, Granada, Spain.}
\cvitem{2013}{Award Top 10 students with highest grades in Granada province - University Access Exam (PAU) (Grade: 13.63 / 14)}

%----------------------------------------------------------------------------------------
%	LANGUAGES SECTION
%----------------------------------------------------------------------------------------

\section{Languages}

\cvitemwithcomment{Spanish}{Mothertongue}{}
\cvitemwithcomment{English}{Cambridge English: Advanced (CAE)}{}

%----------------------------------------------------------------------------------------
%	INTERESTS SECTION
%----------------------------------------------------------------------------------------

\pagebreak

\section{Interests}

\begin{center}
\textit{Topics on which I am interested and some related links.}
\end{center}

\renewcommand{\listitemsymbol}{-~} % Changes the symbol used for lists
\cvlistitem{Mathematics.} 
\cvlistitem{Writing. Mathematics and computer science dissemination. \textcolor{blue}{\href{http://tux.ugr.es/dgiimblog/}{A blog where I write}} (Double Degree in Mathematics and Computer Science's blog). I have written the following posts:
	\begin{itemize}
		\item Algoritmos Genéticos (Genetic Algorithms).
		\item Problemas -- Fibonacci GCD (Problems -- Fibonacci GCD). Written in collaboration with Mario Román.
		\item Teorema de Dini (Dini's theorem).
		\item Segment trees and Range minimum query.
	\end{itemize}}
\cvlistitem{Learning. \textcolor{blue}{\href{https://www.coursera.org/user/i/bfdaf7e4c4a969b9fba10803ab2fb2f0}{My Coursera Profile}}.}
\cvlistitem{Algorithms. \textcolor{blue}{\href{https://www.hackerrank.com/andreshp}{My Hackerrank Profile}}.}
\cvlistitem{Heuristics. \textcolor{blue}{\href{https://github.com/andreshp/TSPSolver}{An Open Source Project}. \href{https://github.com/andreshp/GeneticAlgorithms}{Genetic Algorithms}}.}


%----------------------------------------------------------------------------------------
%	Appendix
%----------------------------------------------------------------------------------------

\section{Appendix: University Subjects Grades}

	\kern 3mm

	\centerline{\textbf{Subjects grades in Computer Science Degree}}
	\kern 2mm
	\centering \begin{tabular}{lcc}
		\toprule
		\textbf{Subject} & \textbf{ Grade } & \textbf{ Qualification}     \\
		\midrule
		Fundamentals of Programming                   & 10   & Cum Laude  \\ 
		Fundamentals of Software                      & 9.6  & Cum Laude  \\
		Fundamentals of Physics and Technologies      & 10   & Cum Laude  \\ 
		Logic and Discrete Methods                    & 10   & Cum Laude  \\
		Programming Methodology                       & 10   & Cum Laude  \\
		Computers Technology                          & 9.6  & Excellent   \\
		Data Structures                               & 10   & Cum Laude  \\
		Computer Structures                           & 9.6  & Cum Laude  \\
		Operating Systems                             & 9.2  & Excellent   \\
		Algorithms                                    & 9.4  & Cum Laude  \\
		Computer Architecture                         & 9.5  & Cum Laude  \\
		Object-Oriented Programming and Design        & 9    & Cum Laude  \\    
		Fundamentals of Data Bases                    & 9.1  & Cum Laude  \\ 
		Fundamentals of Computer Networks             & 7.6  & Good       \\
		Models of Computation                         & 10   & Cum Laude  \\
		Concurrent and Distributed Systems            & 9.3  & Excellent   \\
		Fundamentals of Software Engineering          & 7.5  & Good       \\
		Artificial Intelligence                       & 10   & Cum Laude  \\
		\bottomrule
		\\
	\end{tabular}
	
	\pagebreak
	
	\centerline{\textbf{Subjects Grades in Mathematics Degree}}
	\kern 2mm
	\centering \begin{tabular}{lcc}
		\toprule
		\textbf{Subject} & \textbf{ Grade } & \textbf{ Qualification}   \\ 
		\midrule
		Calculus I                                              & 10  & Cum Laude  \\ 
		Geometry I                                              & 10  & Cum Laude  \\
		Calculus II                                             & 10  & Cum Laude  \\ 
		Geometry II                                             & 10  & Cum Laude  \\
		Descriptive Statistics and Introduction to Probability  & 9.8 & Cum Laude  \\
		Numerical Methods I                                     & 9.8 & Cum Laude  \\
		Analysis I                                              & 9.8 & Cum Laude  \\
		Topology I                                              & 10  & Cum Laude  \\
		Algebra I                                               & 10  & Cum Laude  \\
		Analysis II                                             & 10  & Cum Laude  \\
		Geometry III                                            & 9.8 & Excellent  \\
		Mathematical Models I                                   & 10  & Cum Laude  \\
		Differential Equations I                                & 9   & Excellent  \\
		Probability                                             & 9.5 & Excellent  \\
		Numerical Methods II                                    & 9.5 & Cum Laude  \\
		Complex Analysis                                        & 10  & Cum Laude  \\
		Algebra II                                              & 10  & Cum Laude  \\
		\bottomrule
		\\
	\end{tabular}

%----------------------------------------------------------------------------------------
%	COVER LETTER
%----------------------------------------------------------------------------------------

% To remove the cover letter, comment out this entire block

%\clearpage

%\recipient{HR Department}{Corporation\\123 Pleasant Lane\\12345 City, State} % Letter recipient
%\date{\today} % Letter date
%\opening{Dear Sir or Madam,} % Opening greeting
%\closing{Sincerely yours,} % Closing phrase
%\enclosure[Attached]{curriculum vit\ae{}} % List of enclosed documents

%\makelettertitle % Print letter title

%\lipsum[1-3] % Dummy text

%\makeletterclosing % Print letter signature

%----------------------------------------------------------------------------------------

\end{document}