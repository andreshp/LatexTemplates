\section{The Ising Model}

\subsection{The partition function of the Ising model}

\begin{frame}
  \begin{definition}[The Ising model]
    Let $G = (V, E)$ be a graph. The partition function of the Ising model on $G$ is
    \begin{equation*}
      Z_{\text{Ising}}(G; y, z) = \sum_{\sigma \colon V \to \{0,1\}} y^{m(\sigma)} z^{n_1(\sigma)}.
    \end{equation*}
  \end{definition}
  
  \begin{itemize}
   \item $m(\sigma)$ is the number of monochromatic edges of $\sigma$.
   \item $n_1(\sigma)$ is the number of vertices $v$ with $\sigma(v) = 1$.
  \item $y$ is \emph{the edge interaction}.
  \item $z$ is \emph{the external field}.
  \end{itemize}
  
  \begin{enumerate}
  	\item Setting 1: there is no external field ($z = 1$).  \\
  	We write $Z_{\text{Ising}}(G; y) = Z_{\text{Ising}}(G; y, 1)$.
  	\item Setting 2: there is an external field ($z \ne 1$).  	
  \end{enumerate}
  
  
\end{frame}

\subsection{Example}

\begin{frame}
	\vspace{-4mm}
  \begin{equation*}
    Z_{\text{Ising}}(G; y, z) = \sum_{\sigma \colon V \to \{0,1\}} y^{m(\sigma)} z^{n_1(\sigma)}.
  \end{equation*}
	\vspace{-4mm}
  \begin{example}[The Ising model]
  	\vspace{2mm}
    \begin{columns}
      \column{0.2\textwidth}
      \begin{tikzpicture}
        \begin{scope}[every node/.style={circle,thick,draw}]
          \node (A) at (0,0) {A}; \node (B) at (1,0.5) {B}; \node (C) at (2,0) {C};
        \end{scope}

\begin{scope}[>={Stealth[black]},
              every edge/.style={draw=black,very thick}]
              \path [-] (A) edge node {} (B); \path [-] (A) edge node {} (C); \path [-] (B) edge
              node {} (C);
            \end{scope}
          \end{tikzpicture}
          
          \invisible<1-2>{
          \begin{tikzpicture}
            \begin{scope}[every node/.style={circle,thick,draw}]
              \node[fill=ChetwodeBlue] (A) at (0,0) {A}; \node (B) at (1,0.5) {B}; \node (C) at
              (2,0) {C};
            \end{scope}

\begin{scope}[>={Stealth[black]},
              every edge/.style={draw=black,very thick}]
              \path [-] (A) edge node {} (B); \path [-] (A) edge node {} (C); \path [-] (B) edge
              node {} (C);
            \end{scope}
          \end{tikzpicture}}
          
          \invisible<1-3>{
          \begin{tikzpicture}
            \begin{scope}[every node/.style={circle,thick,draw}]
              \node[fill=ChetwodeBlue] (A) at (0,0) {A}; \node[fill=ChetwodeBlue] (B) at (1,0.5)
              {B}; \node (C) at (2,0) {C};
            \end{scope}

\begin{scope}[>={Stealth[black]},
              every edge/.style={draw=black,very thick}]
              \path [-] (A) edge node {} (B); \path [-] (A) edge node {} (C); \path [-] (B) edge
              node {} (C);
            \end{scope}
          \end{tikzpicture}}
        
          \invisible<1-4>{                  	
          \begin{tikzpicture}
            \begin{scope}[every node/.style={circle,thick,draw}]
              \node[fill=ChetwodeBlue] (A) at (0,0) {A}; \node[fill=ChetwodeBlue] (B) at (1,0.5)
              {B}; \node[fill=ChetwodeBlue] (C) at (2,0) {C};
            \end{scope}

\begin{scope}[>={Stealth[black]},
              every edge/.style={draw=black,very thick}]
              \path [-] (A) edge node {} (B); \path [-] (A) edge node {} (C); \path [-] (B) edge
              node {} (C);
            \end{scope}
          \end{tikzpicture}}
          \vspace{5mm}
          \column{0.6\textwidth}
          \begin{align*}
            \invisible<1>{Z_{\text{Ising}}(G; y, z) & = y^3 + \cdots} \\ \\
            \invisible<1-2>{Z_{\text{Ising}}(G; y, z) & = y^3 + 3yz + \cdots} \\ \\
            \invisible<1-3>{Z_{\text{Ising}}(G; y, z) & = y^3 + 3yz + 3yz^2 + \cdots} \\ \\
            \invisible<1-4>{Z_{\text{Ising}}(G; y, z) & = y^3 + 3yz + 3yz^2 + y^3z^3} \\ \\ 
            \invisible<1-5>{ 
        	Z_{\text{Ising}}(G; y) & = 2 y^3 + 6y
           }                  
          \end{align*}
        \end{columns}
        \invisible<6>{}
        \vspace{-3mm}
      \end{example}
  
 \end{frame}
    \subsection{Computational problems on the Ising model}


\begin{frame}
  \begin{proposition}
    Computing the polynomial $Z_{\text{Ising}}(G; y)$ is $\# \mathsf{P}$-hard.
  \end{proposition}

  \begin{comproblem}[$\textsc{Ising}(y, z)$]
  	\textbf{Instance:} A (multi)graph $G$. \\
  	\textbf{Output:} $Z_{\text{Ising}}(G; y, z)$.
  \end{comproblem}

  \begin{comproblem}[$\textsc{Factor-}K \textsc{-NormIsing}(y, z)$]
  	\textbf{Instance:} A (multi)graph $G$. \\
  	\textbf{Output:} $\,$ A rational number $\hat{N}$ such that
  	\begin{equation*}
  	\frac{1}{K} \hat{N} \le \left| Z_{\text{Ising}}(G; y, z) \right| \le K \hat{N}.
  	\end{equation*}   
  \end{comproblem}

{\color{TurkishRose} $\textsc{Factor-}K \textsc{-NormIsing}(y, z) \le_T \textsc{Ising}(y, z)$.}

	

%  \begin{problem}
%    \textbf{Name:} \ \ \ $\textsc{Distance-}\rho \textsc{-Arg-Ising}(y, z)$. \\
%    \textbf{Instance:} A (multi)graph $G$. \\
%    \textbf{Output:} $\,$ A rational number $\hat{A}$ such that
%    \begin{equation*}
%      \left| \hat{A} - \arg \left( Z_{\text{Ising}}(G; y, z) \right)\right| \le \rho.
%    \end{equation*}   
%  \end{problem}

\end{frame}
