
\section*{Motivation}

\subsection*{Why are we interested in the Ising model?}
%\subsection{The Ising model: a model found in statistical mechanics}

\begin{frame}
	  	The Ising model is a mathematical model of ferromagnetism in statistical mechanics.

    Probability of a configuration $\sigma$ ({\color{TurkishRose}\textit{Boltzmann distribution}}):
    \begin{align*}
    P_\beta(\sigma) & = e^{-\beta H(\sigma)} / Z(G), \\ 
    Z(G) & = \sum_{\sigma \colon V \to \{+,-\}} e^{- \beta H(\sigma)}.
    \end{align*}
    
    \begin{itemize}
    	\item $Z(G)$ is \textit{the partition function of the Ising model}.
    	\item $Z(G)$ encodes information about the physical system.
    \end{itemize}
\end{frame}


\begin{frame}
		     \begin{align*}
		     P_\beta(\sigma) & = e^{-\beta H(\sigma)} / Z(G), \\ 
		     Z(G) & = \sum_{\sigma \colon V \to \{+,-\}} e^{- \beta H(\sigma)}.
		     \end{align*}
		     
	\begin{enumerate}			
		\item {\color{TurkishRose} \textbf{Problem:}} What is the complexity of evaluating $Z(G)$? \\
		%{\color{TurkishRose}\textbf{Answer:}}
		$\# \mathfs{P}$-hard in some cases.
		
		\item For some of those cases, $Z(G)$ admits a \textit{fully polynomial-time randomised approximation scheme} \\
		(Jerrum and Sinclair, 1993,  G\"odel prize in 1996).
		
		\item {\color{TurkishRose} \textbf{Problem:}} What is the complexity of approximating $Z(G)$?
		
		\item {\color{TurkishRose} \textbf{Application:}}  Strongly simulating \textit{instantaneous quantum polynomial-time circuits} is equivalent to approximating $|Z(G)|$ for some parameters.
	\end{enumerate}
\end{frame}

