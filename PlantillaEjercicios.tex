%%%%%%%%%%%%%%%%%%%%%%%%%%%%%%%%%%%%%%%%%%%%%%%%%%%%%%%%%%%%%%%%%%%%%%%%%%%%%
% Plantilla de Latex para trabajos de Matemáticas.
%
% Autor: Andrés Herrera Poyatos (https://github.com/andreshp) 
%         (https://www.researchgate.net/profile/Andres_Poyatos)
%
% Esta plantilla utiliza el paquete Minted para resaltar código.
% Este paquete necesita una instalación previa. Para más información
% ver:http://www.ctan.org/tex-archive/macros/latex/contrib/minted
%
% La plantilla se encuentra adaptada al español.
%
%%%%%%%%%%%%%%%%%%%%%%%%%%%%%%%%%%%%%%%%%%%%%%%%%%%%%%%%%%%%%%%%%%%%%%%%%%%%%

%----------------------------------------------------------------------------------------
%	PAQUETES Y OTRAS CONFIGURACIONES DEL DOCUMENTO
%----------------------------------------------------------------------------------------

\RequirePackage[l2tabu, orthodox]{nag}  % Produce un warnig en caso de usar un comando obsoleto.

\documentclass{article}

% Paquetes para el diseño de página:
\usepackage{fancyhdr}                   % Utilizado para hacer títulos propios.
\usepackage{lastpage}                   % Referencia a la última página. Utilizado para el pie de página.
\usepackage{extramarks}                 % Marcas extras. Utilizado en pie de página y cabecera.
\usepackage[parfill]{parskip}           % Crea una nueva línea entre párrafos.
\usepackage[margin=3cm]{geometry}       % Asigna la "geometría" de las páginas.
\usepackage{lipsum}                     % Texto dummy. Quitar en el documento final.

% Fuente utilizada. Elija uno de ellos:
\usepackage{courier}                    % Fuente Courier.
%\usepackage{fourier}                   % Fuente Adobe Utopia.
\usepackage{microtype}                  % Mejora la letra final de cara al lector.

% Paquetes para imágenes:
\usepackage[usenames,dvipsnames]{color} % Permite crear colores propios. Utilizado para el bg de Minted.
\usepackage{graphicx}                   % Utilizado para insertar gráficos.
\usepackage{layout}                     % Introduce una imagen con el diseño de la página. Quitar en el documento.

% Paquetes para la insercción y el resaltado del código:
\usepackage{minted}                     % Insercción y resaltado de código con Minted.

% Paquetes para matemáticas:                     
\usepackage{amsmath,amsthm,verbatim,amssymb,amsfonts,amscd} % Teoremas, fuentes y símbolos.

% Paquetes para tablas:
\usepackage{booktabs}

% Paquetes para adaptar Látex al Español:
\usepackage[spanish,es-noquoting, es-tabla, es-lcroman]{babel} % Cambia 
\usepackage[utf8]{inputenc}                                    % Permite los acentos.
\selectlanguage{spanish}                                       % Selecciono como lenguaje el Español.

% Estilo de página:
\pagestyle{fancy}                      % fancy
\fancyhf{}                             % Limpia la cabecera y el pie de página.
\geometry{left=3cm,right=3cm,top=3cm,bottom=3cm,headheight=1cm,headsep=0.5cm} % Márgenes y cabecera.

% Espacios en el documento:
\linespread{1.1}                        % Espacio entre líneas.
\setlength\parindent{0pt}               % Selecciona la indentación para cada inicio de párrafo.

% Cabecera del documento:
\renewcommand\headrule{                 % Se ajusta la línea de la cabecera.
\begin{minipage}{1\textwidth}           % Elija una de las siguientes líneas:
%    \hrule width \hsize \kern 1mm \hrule width \hsize height 2pt 
    \hrule width \hsize 
\end{minipage}
}
\lhead{\autor}                          % Parte izquierda.
\chead{}                                % Centro.
\rhead{\titulo}                         % Parte derecha.

% Pie de página del documento:
\renewcommand\footrule{                                 % Se ajusta la línea del pie de página.
\begin{minipage}{1\textwidth}                           % Elija una de las siguientes líneas:
%    \hrule width \hsize height 2pt \kern 1mm \hrule width \hsize   
    \hrule width \hsize   
\end{minipage}\par
}
\lfoot{}                                                 % Parte izquierda.
\cfoot{}                                                 % Centro.
\rfoot{Página\ \thepage\ de\ \protect\pageref{LastPage}} % Parte derecha.

%----------------------------------------------------------------------------------------
%	CONFIGURACIÓN DE LA INCLUSIÓN DE CÓDIGO
%----------------------------------------------------------------------------------------

\usemintedstyle{autumn}                      % Se elige el estilo para minted.
\definecolor{bg}{rgb}{0.95,0.95,0.95}        % Se define el color bg usado para bgcolor de Minted.
\renewcommand\listingscaption{Código}        % Se redefine el nombre dado a un bloque de código.
\renewcommand\listoflistingscaption{Códigos} % Se redefine el nombre dado a la lista de códigos.

%----------------------------------------------------------------------------------------
%	ESTRUCTURA DEL DOCUMENTO
%----------------------------------------------------------------------------------------

\setcounter{secnumdepth}{2}                   % Se enumeran las secciones con profundidad 2.

% Comandos para Matemáticas:
\theoremstyle{plain}
\newtheorem{theorem}{Teorema}
\newtheorem{corollary}{Corolario}
\newtheorem{lemma}{Lema}
\newtheorem{proposition}{Proposición}
\newtheorem*{surfacecor}{Corolario 1}
\newtheorem{conjecture}{Conjetura} 
\newtheorem{question}{Cuestión} 
\theoremstyle{definition}
\newtheorem{definition}{Definición}[section]  % Se crea la subsección "Definición". 

%----------------------------------------------------------------------------------------
%	NUEVOS COMANDOS
%----------------------------------------------------------------------------------------

% Portada:
\newcommand{\titulo}{Título}  % Título del trabajo.
\newcommand{\fecha}{1 \ de \ Enero \ de \ 2015}                         % Fecha.
\newcommand{\asignatura}{Latex}                                         % Asignatura.
\newcommand{\autor}{Andrés Herrera Poyatos}                             % Autor.

%----------------------------------------------------------------------------------------
%	PORTADA 
%----------------------------------------------------------------------------------------

\title{                                             % Título
    \vspace{2in}
    \textmd{\textbf{\asignatura \\ \titulo}} \\         % - Nombre del trabajo
    \normalsize\vspace{0.1in}\small{\fecha}  \\         % - Fecha (Arriba) 
    \vspace{3in}
}

\author{\textbf{\autor}}                            % Autor
\date{}                                             % Fecha. Elija entre esta y la del título.

%----------------------------------------------------------------------------------------

\begin{document}

\maketitle

%----------------------------------------------------------------------------------------
%	ÍNDICE
%----------------------------------------------------------------------------------------

% Profundidad del Índice:
%\setcounter{tocdepth}{1}

\newpage
\tableofcontents
\listoflistings      % Lista con los códigos. Borrar si se desea.
\newpage

%----------------------------------------------------------------------------------------
%	Sección 1: Listing y Minted
%----------------------------------------------------------------------------------------

\begin{section}{Primera Sección}

\begin{definition}
El paquete \textbf{Minted} permite insertar código en el texto de diferentes formas.
Utiliza para ello \textbf{pygments}. Este se debe instalar por separado.
\end{definition}

\begin{subsection}{Listing con Minted}

\lipsum[1]

Puedo referenciar el Código \ref{listing:page}! 

\begin{listing}
\caption{Estructura que representa una Página en Linux}
\begin{minted}[bgcolor=bg]{c}
#include <linux/mm_types.h>

struct page {
   unsigned long flags;
   atomic_t _count;
   atomic_t _mapcount;
   unsigned long private;
   struct address_space *mapping;
   pgoff_t index;
   struct list_head lru;
   void *virtual;
}
\end{minted}
\label{listing:page}
\end{listing}

\lipsum[2]

\end{subsection}

\begin{subsection}{Minted en línea}

\lipsum[3]

\mint[bgcolor=bg]{c}|struct page * alloc_pages(gfp_t gfp_mask, unsigned int order)|

\lipsum[4]

\end{subsection}

\end{section}

%----------------------------------------------------------------------------------------
%	Sección 2: Minted en línea
%----------------------------------------------------------------------------------------
\begin{section}{Segunda Sección}

\begin{subsection}{Tablas}

\lipsum[5]

\begin{table}[ht]
\caption{Zonas de Memoria en la arquitectura x86-32}
\kern 1mm                                  % Separa caption de la tabla
\centering \begin{tabular}{@{}llc@{}}
\toprule
\multicolumn{2}{c}{Info}                                            \\
\cmidrule(r){1-2}
\textbf{Zona} & \textbf{Descripción}   & \textbf{Memoria Física}    \\ 
\midrule
ZONE\_DMA     & Soporta DMA            & \textless \ 16MB           \\ 
ZONE\_NORMAL  & Páginas normales       & 16–896MB                   \\
ZONE\_HIGHMEM & Alta memoria           & \textgreater \ 896MB       \\ 
\bottomrule
\end{tabular}
\label{table:memoria-x86-32}
\end{table}

\end{subsection}

\begin{subsection}{Descripción}

\begin{description}
\item[ZONE\_DNA] Contiene páginas que soportan la entrada y salida de datos mediante DMA.
\item[ZONE\_DNA\_32] Es análoga a la anterior pero ls páginas solo son accesibles para
dispositivos de 32-bits.
\item[ZONE\_NORMAL] Esta zona contiene páginas normales y regularmente mapeadas.
\item[ZONE\_HIGHMEM] Esta zona contiene \textit{alta memoria}, que consiste en
 aquellas páginas que son mapeadas con poca frecuencia.
\end{description}

\end{subsection}

\pagebreak

\begin{subsection}{Configuración de la página}
Se muestra la configuración de la página. Eliminarla en el documento final. 
Puede dar problemas con la cabecera y pie de página.

\centering\layout
\end{subsection}

\end{section}

\end{document}